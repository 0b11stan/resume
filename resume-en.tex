\RequirePackage{snapshot}
\documentclass[10pt,a4paper,ragged2e,withhyper]{altacv}
\geometry{left=1.25cm,right=1.25cm,top=1cm,bottom=1cm,columnsep=1.2cm}
\usepackage{paracol}
\usepackage{fancybox}
\usepackage[rm]{roboto}
\usepackage[defaultsans]{lato}
% \usepackage{sourcesanspro}
\renewcommand{\familydefault}{\sfdefault}

% Change the colours if you want to
\definecolor{SlateGrey}{HTML}{2E2E2E}
\definecolor{LightGrey}{HTML}{666666}
\definecolor{DarkPastelRed}{HTML}{2c2f33}
\definecolor{PastelRed}{HTML}{7289da}
\definecolor{GoldenEarth}{HTML}{99aab5}
\colorlet{name}{black}
\colorlet{tagline}{PastelRed}
\colorlet{heading}{DarkPastelRed}
\colorlet{headingrule}{GoldenEarth}
\colorlet{subheading}{PastelRed}
\colorlet{accent}{PastelRed}
\colorlet{emphasis}{SlateGrey}
\colorlet{body}{LightGrey}

% Change some fonts, if necessary
\renewcommand{\namefont}{\Huge\rmfamily\bfseries}
\renewcommand{\personalinfofont}{\footnotesize}
\renewcommand{\cvsectionfont}{\LARGE\rmfamily\bfseries}
\renewcommand{\cvsubsectionfont}{\large\bfseries}

% Change the bullets for itemize and rating marker
% for \cvskill if you want to
\renewcommand{\itemmarker}{{\small\textbullet}}
\renewcommand{\ratingmarker}{\faCircle}

%% Use (and optionally edit if necessary) this .cfg if you
%% want to use an author-year reference style like APA(6)
%% for your publication list
% When using APA6 if you need more author names to be listed
% because you're e.g. the 12th author, add apamaxprtauth=12
\usepackage[backend=biber,style=apa6,sorting=ydnt]{biblatex}
\defbibheading{pubtype}{\cvsubsection{#1}}
\renewcommand{\bibsetup}{\vspace*{-\baselineskip}}
\AtEveryBibitem{\makebox[\bibhang][l]{\itemmarker}}
\setlength{\bibitemsep}{0.25\baselineskip}
\setlength{\bibhang}{1.25em}


%% Use (and optionally edit if necessary) this .cfg if you
%% want an originally numerical reference style like IEEE
%% for your publication list
% \usepackage[backend=biber,style=ieee,sorting=ydnt]{biblatex}
%% For removing numbering entirely when using a numeric style
\setlength{\bibhang}{1.25em}
\DeclareFieldFormat{labelnumberwidth}{\makebox[\bibhang][l]{\itemmarker}}
\setlength{\biblabelsep}{0pt}
\defbibheading{pubtype}{\cvsubsection{#1}}
\renewcommand{\bibsetup}{\vspace*{-\baselineskip}}


%% sample.bib contains your publications
\addbibresource{sample.bib}

\begin{document}
\name{Tristan Auvinet Pinaudeau}
\tagline{Cybersecurity Engineer}
%% You can add multiple photos on the left or right
\photoR{2.8cm}{face}
% \photoL{2.5cm}{Yacht_High,Suitcase_High}

\personalinfo{%
  % Not all of these are required!
  \email{tristan@tic.sh}
  \phone{+33 6 81 67 53 37}
  \mailaddress{164 rue d'Ornano, 33000 Bordeaux, France}
  \\
  \homepage{blog.tic.sh}
  \twitter{@0b11stan}
  \linkedin{tristan-pinaudeau}
  \github{0b11stan}
  %\orcid{0000-0000-0000-0000}
  %% You can add your own arbitrary detail with
  %% \printinfo{symbol}{detail}[optional hyperlink prefix]
  % \printinfo{\faPaw}{Hey ho!}[https://example.com/]
  %% Or you can declare your own field with
  %% \NewInfoFiled{fieldname}{symbol}[optional hyperlink prefix] and use it:
  % \NewInfoField{gitlab}{\faGitlab}[https://gitlab.com/]
  % \gitlab{your_id}
  %%
  %% For services and platforms like Mastodon where there isn't a
  %% straightforward relation between the user ID/nickname and the hyperlink,
  %% you can use \printinfo directly e.g.
  % \printinfo{\faMastodon}{@username@instace}[https://instance.url/@username]
  %% But if you absolutely want to create new dedicated info fields for
  %% such platforms, then use \NewInfoField* with a star:
  % \NewInfoField*{mastodon}{\faMastodon}
  %% then you can use \mastodon, with TWO arguments where the 2nd argument is
  %% the full hyperlink.
  % \mastodon{@username@instance}{https://instance.url/@username}
}

\makecvheader
%% Depending on your tastes, you may want to make fonts of itemize environments slightly smaller
% \AtBeginEnvironment{itemize}{\small}

%% Set the left/right column width ratio to 6:4.
\columnratio{0.6}

% Start a 2-column paracol. Both the left and right columns will automatically
% break across pages if things get too long.
\begin{paracol}{2}
\cvsection{Work Experience}

\cvevent{Cybersecurity Consultant}{Capgemini}{October 2021 -- December 2024}{Bordeaux (Merignac)}
\begin{itemize}
  \item Long lasting Red and Purple Teaming (banking and aerospace)
  \item External and Internal pentests (public, banking and care sector)
  \item Full build of a complex air-gapped network (aerospace sector)
  \item Automation of OT network maintenance (aerospace sector)
  \item Teaching of OS (kernel) security for Master's students
  \item WAF solution audit for CSPN certification (ANSSI)
  \item Semi-pro CTF player
\end{itemize}

\divider

\cvevent{Site Reliability Engineer}{Cdiscount}{Sept. 2018 -- Sept. 2021}{Bordeaux (Baccalan)}
\begin{itemize}
  \item Integration of a highly available Hashicorp Vault
  \item Integration of a resilient observability platform
  \item Maintenance of messaging and infrastructure platforms
\end{itemize}

\divider

\cvevent{DevOps Developer}{HelloAsso}{Sept. 2017 -- Sept. 2018}{Bordeaux (Darwin)}
\begin{itemize}
  \item Development of a compliance analysis system
  \item N+1 maintenance and support for .NET and Azure environments
\end{itemize}

\divider

\cvevent{Full Stack Developer}{Touton SA}{May 2016 -- July 2017}{Bordeaux (Baccalan)}
\begin{itemize}
  \item Converting a legacy application (Access) to ASP .NET
\end{itemize}

\medskip

\cvsection{Other Activities}

  \cvevent{}{Hackathons / CTFs}{}{}
  Numerous CTFs and Hackathons either as a hobby or as part of the official \href{https://www.capgemini.com/fr-fr/perspectives/publications/les-aces-of-spades-lequipe-de-capture-the-flag-de-capgemini/}{\bf{Aces of Spades}} team.

  \divider

  \cvevent{}{Conferences and blog posts}{}{}
  \begin{itemize}
    \item Occasional blog posts on \href{https://blog.tic.sh}{\bf{https://blog.tic.sh/}}
    \item \href{https://youtu.be/GpJdcgxwxVE?t=23867}{\bf{Conference on NixOS}} at the Hack'It'N 2022 convention.
  \end{itemize}

  \divider

  \cvevent{}{Development}{}{}
  \begin{itemize}
    \item Minor open-source contributions \\
      (\href{https://github.com/milesmcc/shynet/pull/236}{\bf{shynet}}, \href{https://github.com/NixOS/nixpkgs/pull/259785}{\bf{nixpkgs}}, molecule \href{https://github.com/ansible-community/molecule-podman/pull/83}{\bf{PR\#83}} and \href{https://github.com/ansible-community/molecule-podman/pull/79}{\bf{PR\#79}})
    \item Tool development for personal usage.
  \end{itemize}



%  \begin{itemize}
%    \item Numerous CTFs and Hackathons either as a hobby or as part of the official \href{https://www.capgemini.com/fr-fr/perspectives/publications/les-aces-of-spades-lequipe-de-capture-the-flag-de-capgemini/}{\bf{Aces of Spades}} team.
%    \item Occasional blog posts and conferences about cybersecurity.
%    \item Some spare time tinkering with NixOS and other exciting techs.
%    \item Minor open-source contributions (\href{https://github.com/milesmcc/shynet/pull/236}{\bf{shynet}}, \href{https://github.com/NixOS/nixpkgs/pull/259785}{\bf{nixpkgs}}, molecule-podman \href{https://github.com/ansible-community/molecule-podman/pull/83}{\bf{PR\#83}} and \href{https://github.com/ansible-community/molecule-podman/pull/79}{\bf{PR\#79}})
%    \item Tool development for personal usage.
%  \end{itemize}

\medskip

\switchcolumn

\cvsection{Strengths}

  \cvtag{GNU/Linux}
  \cvtag{Containerization}
  \cvtag{Python}

  \smallskip

  \cvtag{Pentest}
  \cvtag{Qemu/KVM}
  \cvtag{Ansible}
  \cvtag{RedHat}

  \smallskip

  \cvtag{Enthusiastic}
  \cvtag{Curious}
  \cvtag{Love to share}

\cvsection{Languages}

  \cvskill{French}{5}
  \divider

  \cvskill{English (TOEIC 960/990)}{4}
  \divider

  \cvskill{German}{1}
  \divider

\cvsection{Formations}

  \cvevent{Specialized Master in Cybersecurity}{ENSEEIHT Toulouse}{Sept. 2020 -- Sept. 2021}{}
  An highly technical course on low-level computer hacking, the result of collaboration between INSA, ENAC and ENSEEIHT. (c.f. \href{https://tls-sec.github.io/documents/Syllabus_Ms_securite_Informatique_V1.pdf}{\bf{syllabus}}).

  \divider

  \cvevent{Level 1 RNCP qualification}{EPSI Bordeaux}{Sept 2018 -- Sept 2020}{}
  Obtained upon the completion of a master's degree in general computer science (cybersecurity speciality) under a sandwich course at Cdiscount.

  \divider

  \cvevent{Level 2 RNCP qualification}{EPSI Bordeaux}{Sept 2017 -- Sept 2018}{}
  Obtained upon the completion of a bachelor's degree in general computer science under a sandwich course at HelloAsso (Valedictorian).

  \divider

  \cvevent{BTS SIO SLAM}{EPSI Bordeaux}{Sept 2015 -- Sept 2017}{}

% \divider

\end{paracol}


\end{document}
