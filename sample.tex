%%%%%%%%%%%%%%%%%
% This is an sample CV template created using altacv.cls
% (v1.6.4, 13 Nov 2021) written by LianTze Lim (liantze@gmail.com). Now compiles with pdfLaTeX, XeLaTeX and LuaLaTeX.
%
%% It may be distributed and/or modified under the
%% conditions of the LaTeX Project Public License, either version 1.3
%% of this license or (at your option) any later version.
%% The latest version of this license is in
%%    http://www.latex-project.org/lppl.txt
%% and version 1.3 or later is part of all distributions of LaTeX
%% version 2003/12/01 or later.
%%%%%%%%%%%%%%%%

%% Use the "normalphoto" option if you want a normal photo instead of cropped to a circle
% \documentclass[10pt,a4paper,normalphoto]{altacv}

\documentclass[10pt,a4paper,ragged2e,withhyper]{altacv}
%% AltaCV uses the fontawesome5 and packages.
%% See http://texdoc.net/pkg/fontawesome5 for full list of symbols.

% Change the page layout if you need to
\geometry{left=1.25cm,right=1.25cm,top=1.25cm,bottom=1.25cm,columnsep=1.2cm}

% The paracol package lets you typeset columns of text in parallel
\usepackage{paracol}
\usepackage{fancybox}

% Change the font if you want to, depending on whether
% you're using pdflatex or xelatex/lualatex
\ifxetexorluatex
  % If using xelatex or lualatex:
  \setmainfont{Roboto Slab}
  \setsansfont{Lato}
  \renewcommand{\familydefault}{\sfdefault}
\else
  % If using pdflatex:
  \usepackage[rm]{roboto}
  \usepackage[defaultsans]{lato}
  % \usepackage{sourcesanspro}
  \renewcommand{\familydefault}{\sfdefault}
\fi

% Change the colours if you want to
\definecolor{SlateGrey}{HTML}{2E2E2E}
\definecolor{LightGrey}{HTML}{666666}
\definecolor{DarkPastelRed}{HTML}{2c2f33}
\definecolor{PastelRed}{HTML}{7289da}
\definecolor{GoldenEarth}{HTML}{99aab5}
\colorlet{name}{black}
\colorlet{tagline}{PastelRed}
\colorlet{heading}{DarkPastelRed}
\colorlet{headingrule}{GoldenEarth}
\colorlet{subheading}{PastelRed}
\colorlet{accent}{PastelRed}
\colorlet{emphasis}{SlateGrey}
\colorlet{body}{LightGrey}

% Change some fonts, if necessary
\renewcommand{\namefont}{\Huge\rmfamily\bfseries}
\renewcommand{\personalinfofont}{\footnotesize}
\renewcommand{\cvsectionfont}{\LARGE\rmfamily\bfseries}
\renewcommand{\cvsubsectionfont}{\large\bfseries}


% Change the bullets for itemize and rating marker
% for \cvskill if you want to
\renewcommand{\itemmarker}{{\small\textbullet}}
\renewcommand{\ratingmarker}{\faCircle}

%% Use (and optionally edit if necessary) this .cfg if you
%% want to use an author-year reference style like APA(6)
%% for your publication list
% When using APA6 if you need more author names to be listed
% because you're e.g. the 12th author, add apamaxprtauth=12
\usepackage[backend=biber,style=apa6,sorting=ydnt]{biblatex}
\defbibheading{pubtype}{\cvsubsection{#1}}
\renewcommand{\bibsetup}{\vspace*{-\baselineskip}}
\AtEveryBibitem{\makebox[\bibhang][l]{\itemmarker}}
\setlength{\bibitemsep}{0.25\baselineskip}
\setlength{\bibhang}{1.25em}


%% Use (and optionally edit if necessary) this .cfg if you
%% want an originally numerical reference style like IEEE
%% for your publication list
% \usepackage[backend=biber,style=ieee,sorting=ydnt]{biblatex}
%% For removing numbering entirely when using a numeric style
\setlength{\bibhang}{1.25em}
\DeclareFieldFormat{labelnumberwidth}{\makebox[\bibhang][l]{\itemmarker}}
\setlength{\biblabelsep}{0pt}
\defbibheading{pubtype}{\cvsubsection{#1}}
\renewcommand{\bibsetup}{\vspace*{-\baselineskip}}


%% sample.bib contains your publications
\addbibresource{sample.bib}

\begin{document}
\name{Tristan Pinaudeau}
\tagline{Consultant en sécurité offensive éthique}
%% You can add multiple photos on the left or right
\photoR{2.8cm}{face}
% \photoL{2.5cm}{Yacht_High,Suitcase_High}

\personalinfo{%
  % Not all of these are required!
  \email{tristan@tic.sh}
  \phone{+33 6 81 67 53 37}
  \mailaddress{164 rue d'Ornano, 33000 Bordeaux, France}
  \homepage{tic.sh}
  \twitter{@0b11stan}
  \linkedin{tristan-pinaudeau}
  \github{0b11stan}
  %\orcid{0000-0000-0000-0000}
  %% You can add your own arbitrary detail with
  %% \printinfo{symbol}{detail}[optional hyperlink prefix]
  % \printinfo{\faPaw}{Hey ho!}[https://example.com/]
  %% Or you can declare your own field with
  %% \NewInfoFiled{fieldname}{symbol}[optional hyperlink prefix] and use it:
  % \NewInfoField{gitlab}{\faGitlab}[https://gitlab.com/]
  % \gitlab{your_id}
  %%
  %% For services and platforms like Mastodon where there isn't a
  %% straightforward relation between the user ID/nickname and the hyperlink,
  %% you can use \printinfo directly e.g.
  % \printinfo{\faMastodon}{@username@instace}[https://instance.url/@username]
  %% But if you absolutely want to create new dedicated info fields for
  %% such platforms, then use \NewInfoField* with a star:
  % \NewInfoField*{mastodon}{\faMastodon}
  %% then you can use \mastodon, with TWO arguments where the 2nd argument is
  %% the full hyperlink.
  % \mastodon{@username@instance}{https://instance.url/@username}
}

\makecvheader
%% Depending on your tastes, you may want to make fonts of itemize environments slightly smaller
% \AtBeginEnvironment{itemize}{\small}

%% Set the left/right column width ratio to 6:4.
\columnratio{0.6}

% Start a 2-column paracol. Both the left and right columns will automatically
% break across pages if things get too long.
\begin{paracol}{2}
\cvsection{Expériences}

\cvevent{Consultant cybersécurité}{Capgemini}{Octobre 2021 -- Aujourd'hui}{Pessac, Bordeaux}
\begin{itemize}
\item Audit d'une solution WAF pour une certification CSPN
\item Tests d'intrusion externe et interne
\end{itemize}

\divider

  \cvevent{Ingénieur SRE}{Cdiscount}{Sept. 2018 -- Sept. 2021}{Bordeaux (Baccalan)}
\begin{itemize}
\item Intégration d'une solution de coffre-fort numérique résiliente
\item Intégration d'une solution d'observabilité résiliente
\item Maintenance de plateformes de messaging et d'infrastructure
\end{itemize}

\divider

\cvevent{Développeur DevOps}{HelloAsso}{Sept. 2017 -- Sept. 2018}{Bordeaux (Darwin)}
\begin{itemize}
  \item Développement d'un système d'analyse de la conformité
  \item Maintenance et support N+1 environnement .NET et Azure
\end{itemize}

\divider

\cvevent{Développeur Full Stack}{Touton SA}{Mai 2016 -- Juillet 2017}{Bordeaux (Baccalan)}
\begin{itemize}
  \item Passage d'une application legacy (Access) en ASP .NET
\end{itemize}

\cvsection{Activités}

  \cvevent{}{Hackathons / CTFs}{}{}
  \begin{itemize}
    \item THC CTF (57/374) -- 2021: binaire, crypto, web, android (\href{https://blog.tic.sh/thc-2021-mission-impossible/}{\color{PastelRed}writeup})
    \item Google Code Jam (Qualif + Round1) -- 2021: maths, algo
    \item Sunshine CTF (72/750) -- 2020: binaire, web
    \item Square CTF (127/610) -- 2020: binaire, crypto, web
    \item StartupWeekendBdx (2/40) -- 2019: gestion, marketing, finance
    \item Sthack Conf \& CTF -- 2019: binaire, IA, web, network
    \item HackBordeaux (1\textsuperscript{er} au défi Ekino) -- 2019: Android, BLE
    \item HackBordeaux (1\textsuperscript{er} au défi Lectra) -- 2018: Graphql, React
  \end{itemize}

  \cvevent{}{Projets}{}{}
  \begin{itemize}
    \item Martine (WIP): Site de partage de photos
      \begin{itemize}
        \item[>] Python (Flask) -- Go -- VueJs -- Docker -- Ansible
      \end{itemize}
    \item AKO (2019): Application innovante de partage de compétences.
      \begin{itemize}
        \item[>] Python -- Graphql -- Hasura -- React
      \end{itemize}
    \item Heimdal (2018): Développement d'un service HIDS déporté.
      \begin{itemize}
        \item[>] Python -- Bash -- Docker
      \end{itemize}
  \end{itemize}

\medskip

%% Switch to the right column. This will now automatically move to the second
%% page if the content is too long.
\switchcolumn

\cvsection{Forces}

  \cvtag{GNU/Linux}
  \cvtag{Python}
  \cvtag{Docker / Podman}
  \cvtag{Ansible}
  \cvtag{Qemu/KVM}
  \cvtag{Tests Auto}
  \cvtag{Tests d'intrusion}

  \divider\smallskip

  \cvtag{Passionné}
  \cvtag{Pédagogue}
  \cvtag{Curieux}

\cvsection{Langues}

  \cvskill{Français}{5}
  \divider

  \cvskill{Anglais (TOEIC 960/990)}{4}
  \divider

%% Yeah I didn't spend too much time making all the
%% spacing consistent... sorry. Use \smallskip, \medskip,
%% \bigskip, \vspace etc to make adjustments.
\medskip

\cvsection{Formations}

  \cvevent{Master Spécialisé en Sécurité Informatique}{ENSEEIHT}{Sept. 2020 -- Sept. 2021}{}
  Une formation de haut niveau très technique, résultat d'une collaboration entre
  l'INSA, l'ENAC et l'ENSEEIHT. (c.f. \href{https://tls-sec.github.io/documents/Syllabus_Ms_securite_Informatique_V1.pdf}{syllabus})

  \divider

  \cvevent{Titre RNCP de niveau 1}{EPSI Bordeaux}{Sept 2018 -- Sept 2020}{}
  Obtenu à la fin d'un cycle d'ingénierie en informatique généraliste (spécialité
  cybersécurité) réalisé en alternance à Cdiscount.

  \divider

  \cvevent{Titre RNCP de niveau 2}{EPSI Bordeaux}{Sept 2017 -- Sept 2018}{}
  Obtenu à la fin d'un cycle bachelor en informatique généraliste réalisé en
  alternance.
  (Major de Promotion)

  \divider

  \cvevent{BTS SIO SLAM}{EPSI Bordeaux}{Sept 2015 -- Sept 2017}{}

% \divider

\end{paracol}


\end{document}
